\documentclass[a4paper, 12pt, oneside]{article}

% document formatting
\usepackage[margin=0.75in]{geometry}

% Here we put the packages to use
\usepackage[english]{babel}
\usepackage{graphics}
\usepackage{comment}

% define personal colors
\usepackage{xcolor}
\definecolor{commentsColor}{rgb}{0.497495, 0.497587, 0.497464}
\definecolor{keywordsColor}{rgb}{0.000000, 0.000000, 0.635294}
\definecolor{stringColor}{rgb}{0.558215, 0.000000, 0.135316}

% Code formatting packages
\usepackage{listings}
\lstset{
    backgroundcolor=\color{white},   % choose the background color; you must add \usepackage{color} or \usepackage{xcolor}
    basicstyle=\footnotesize,        % the size of the fonts that are used for the code
    breakatwhitespace=false,         % sets if automatic breaks should only happen at whitespace
    breaklines=true,                 % sets automatic line breaking
    captionpos=b,                    % sets the caption-position to bottom
    commentstyle=\color{commentsColor}\textit,    % comment style
    deletekeywords={...},            % if you want to delete keywords from the given language
    escapeinside={\%*}{*)},          % if you want to add LaTeX within your code
    extendedchars=true,              % lets you use non-ASCII characters; for 8-bits encodings only, does not work with UTF-8
    frame=tb,	                   	   % adds a frame around the code
    keepspaces=true,                 % keeps spaces in text, useful for keeping indentation of code (possibly needs columns=flexible)
    keywordstyle=\color{keywordsColor}\bfseries,       % keyword style
    language=Python,                 % the language of the code (can be overrided per snippet)
    otherkeywords={*,...},           % if you want to add more keywords to the set
    numbers=left,                    % where to put the line-numbers; possible values are (none, left, right)
    numbersep=5pt,                   % how far the line-numbers are from the code
    numberstyle=\tiny\color{commentsColor}, % the style that is used for the line-numbers
    rulecolor=\color{black},         % if not set, the frame-color may be changed on line-breaks within not-black text (e.g. comments (green here))
    showspaces=false,                % show spaces everywhere adding particular underscores; it overrides 'showstringspaces'
    showstringspaces=false,          % underline spaces within strings only
    showtabs=false,                  % show tabs within strings adding particular underscores
    stepnumber=1,                    % the step between two line-numbers. If it's 1, each line will be numbered
    stringstyle=\color{stringColor}, % string literal style
    tabsize=2,	                     % sets default tabsize to 2 spaces
    title=\lstname,                  % show the filename of files included with \lstinputlisting; also try caption instead of title
    columns=fixed                    % Using fixed column width (for e.g. nice alignment)
}

% link formatting
\usepackage{hyperref}
\hypersetup{
    colorlinks=true,
    linkcolor=orange,
    filecolor=magenta,      
    urlcolor=blue,
}





\title{First \LaTeX\ Document }
\author{Florian Rascoussier}
\date{April 19, 2021}

\begin{document}
    \maketitle

    \section{Introduction}
    \LaTeX\ is a very powerful tool to create documents or scientific
    papers. A good introduction tutorial can be found 
    \href{https://www.overleaf.com/learn/latex/Main_Page}{here | Overleaf doc}

    \section{Once upon a time}
    Once upon a time, there were an Hello World...

\begin{comment}
    This is a multiline comment 
    Useful to, for instance, comment out 
    slow-rendering parts
    while working on a draft.
\end{comment}

    \section{Marvels incoming}
    Hello World...

    \section{Code time}
    Here is some typical Python code.

    \begin{lstlisting}[language=Python, caption=Python example, frame=single]
    # Program to display the Fibonacci sequence up to n-th term

    nterms = int(input("How many terms? "))

    # first two terms
    n1, n2 = 0, 1
    count = 0

    # check if the number of terms is valid
    if nterms <= 0:
    print("Please enter a positive integer")
    elif nterms == 1:
    print("Fibonacci sequence upto",nterms,":")
    print(n1)
    else:
    print("Fibonacci sequence:")
    while count < nterms:
        print(n1)
        nth = n1 + n2
        # update values
        n1 = n2
        n2 = nth
        count += 1
    \end{lstlisting}
    
    \begin{lstlisting}[language=bash, caption=Python example output, frame=single]
    (base) onyr@aezyr:~/Documents/code/python/python_intro$ python3 fibonacci_sequence.py 
    How many terms? 10
    Fibonacci sequence:
    0
    1
    1
    2
    3
    5
    8
    13
    21
    34
    \end{lstlisting}

    And here is some from a file 

    \lstinputlisting[language=C, caption=Python example output, frame=single]{src_code/check_leap_year.c}

\end{document}

