\documentclass{article}

\usepackage[french]{babel}
\usepackage[T1]{fontenc}
\usepackage[a4paper]{geometry}

\usepackage{csquotes}
\usepackage[
backend=biber,
style=numeric,
sorting=ynt
]{biblatex}
\addbibresource{biblio.bib}

\usepackage{epigraph}
\usepackage{dirtytalk}

\title{Cambridge Analytica – Données personnelles et guerre psychologique au service des puissants}


\begin{document}

\begin{titlepage}
    \begin{center}
        \vspace*{1cm}
        \Huge
        \textbf{Société Numérique}
        
        \vspace{2cm}
        
        \Huge
        \textbf{Cambridge Analytica – Données personnelles et guerre psychologique au service des puissants}
            
        \vspace{2cm}
        \Huge
        Florian Rascoussier
        
        \vspace{1.5cm}
        \LARGE
        27-28 janvier 2022
        
        \vfill
        
        \LARGE
        Les données personnelles dans l'ère du numérique\\
        Rapport d'exposé\\
        INSA Lyon\\
        4IF\\
            
    \end{center}
\end{titlepage}


\textbf{{\LARGE Cambridge Analytica – Données personnelles et guerre psychologique au service des puissants}}\\

\epigraph{La connaissance est le pouvoir. L'unique.}{\textit{Lionelle Nugon-Baudon}}

\normalsize
Dès 2006, le mathématicien Clive Humby l’énonce : \say{Les données sont le nouveau pétrole} (\say{Data is the new oil}) \cite{Haupt2016-nl, Joris_Toonders2014-lb, noauthor_undated-nq}. Il ne se contente pas de cette simple phrase iconique et va plus loin dans ses explications. Pour lui, les données représente une ressource brute à la valeur intrinsèque. Cependant, de même que le pétrole à besoin d’être raffiné pour être exploitable, il convient de faire de même avec les données. Ensuite, à la manière du pétrole qui peut être utilisé pour fabriquer des produits utiles et divers comme l’essence ou le plastique, les données peuvent ainsi être utilisées à des fins multiples, notamment pour entraîner des algorithmes d’IA et créer des modèles descriptifs et prédictifs.\\

Le terme anglais pour les données est le bien connu “Data”, du latin datum : “Qui est donné”, et qui représente selon le dictionnaire Le Robert “des données numériques” \cite{noauthor_undated-cy}. Aujourd’hui, les données sont en effet majoritairement numérisées, et le terme est ainsi réapparu spécifiquement pour l’informatique en 1954 \cite{Marr2018-dd}. Aujourd’hui les chiffres donnent le tournis. En 2018, Bernard Marr indiquait, pour Forbes, qu’il y avait déjà 2.5 trillions d’octets de données générées chaque jour \cite{Marr2018-dd}. Selon le site Statista, il y avait 4,66 milliards d’utilisateur actifs sur internet en janvier 2021, dont 4,2 milliards d’utilisateurs de réseaux sociaux \cite{noauthor_undated-vd}.\\

Toutes ces données ont une réelle valeur monétaire. Par exemple, une donnée médicale susceptible de renseigner l’état de santé d’une personne peut valoir plusieurs dizaines de dollars \cite{Benyayer2017-xt}. Au total, le marché des données représenterait à lui tout seul 3 billions de dollars en 2017 selon un article du World Economic Forum \cite{noauthor_undated-yi}, qui insiste sur le fait que l’enjeu actuel devient celui de la propriété des données. Parmi ces données, se trouvent les informations personnelles et souvent confidentielles de milliards d’utilisateurs du web. Le recoupement et l’exploitation de ces masses de données permettent ainsi de dresser le profil psychologique complet de millions de personnes et de groupes entiers.\\

Outre les problèmes de vie privée, ces masses de données peuvent donc être utilisées à des fins de prédictions comportementale qui sont les bases d’un nouveau genre d’influence des populations. Comme le dit Alexander Nix, CEO de Cambridge Analytica, \say{Aujourd'hui, aux États-Unis, nous avons quelque part près de quatre ou cinq mille points de données sur chaque individu... Nous modélisons donc la personnalité de chaque adulte à travers les États-Unis, quelque 230 millions de personnes.} \cite{Cheshire2016-xi}. Cambridge Analytica Limited (CA) était une filiale du groupe Strategic Communication Laboratories (SCL), un groupement britannique spécialisé dans la communication, ciblant notamment les élections aux moyens de techniques de guerre psychologique utilisées par l’entreprise mère depuis la guerre d’Iraq pour le compte des américains et des britanniques \cite{Barry2018-zs}.\\

Cambridge Analytica est donc une entreprise fondé sur une vision : celle d’hommes puissants et fortunés, proches du pouvoir et qui cherchent à étendre leur influence et celles de leurs politiciens proches au moyens de divers techniques issues du croisement entre psychologie, Big Data, et manipulation de masse. L’entreprise voit le jour en 2013, du croisement entre Steve Bannon, militant conservateur américain proche de l’extrême droite qui deviendra directeur exécutif de la campagne de Donald Trump en 2016 ; Robert Mercer, milliardaire américain gestionnaire de hedge fund : et surtout Alexander Nix, homme d’affaire britannique et co-directeur de Cambridge Analytica. L’objectif est d’emblé celui de la guerre psychologique sous une nouvelle approche expérimentale, selon les propos même de Bannon qui investit 15 millions de dollars aux côtés de Mercer dans l'entreprise \cite{Silver2018-rq}.\\

La guerre psychologique (PSYWAR) désigne un type de conflit, un ensemble de pratiques et méthodes qui visent à manipuler un groupe ennemi afin de le démoraliser, de le pousser à la rébellion, à l’inaction ou à la fuite \cite{Chaliand1992-mp}. Le concept n’est pas nouveau, puisqu’on en retrouve des traces dès l’antiquité. Il en est notamment fait mention dans l’ouvrage L’art de la guerre de Sun Tzu, probablement écrit au Vème siècle avant notre ère : \say{Toute guerre est fondée sur la tromperie.} \cite{Tzu_undated-du}, même si le terme est plus récent. Aujourd’hui, les masses de données et les avancées en matière de manipulation permettent la mise en place d’un nouveau genre d’influences qui pose question.\\

Ainsi, Cambridge Analytica est passée sous l’oeil des projecteurs médiatique grâce à Christopher Wylie, ancien employé de SCL et CA. Son travail ainsi que les révélations de journaux internationaux tels que le New York Times ou The Guardian, ont permit de révéler au monde l’ampleur du scandale. La firme a ainsi eu un rôle important dans la scène politique internationale, aux États Unis et dans le reste du monde. Dès 2015, l’entreprise et ses filiales sont mentionnées comme ayant largement participé à la campagne de Ted Cruz \cite{Davies2015-wq}, pour un montant de 6 millions de dollars.\\

C’est surtout pour son action durant la campagne de Donald Trump de 2016 que l’entreprise est pointée du doigt. Durant celle-ci, l’entreprise, grâce à ses différentes filiales tel qu’AggregateIQ à pu collecter 87 millions de profils Facebook \cite{noauthor_undated-yf}. Pour y parvenir, les agens de CA et ses filiales sont passés par le Docteur Aleksandr Kogan, de l’université de Cambridge et chercheur en psychologie et profilage qui avait déjà l’habitude de travailler avec les données de Facebook. En utilisant un quiz de personnalité et divers autres questionnaires parfois rémunérés, ils ont pu demander aux utilisateurs de leur donner accès de manière légale aux données de leurs comptes Facebook. De manière illégale, ils pouvaient également récupérer des informations sur leur contacts et ainsi relier un grand nombre de personnes \cite{Cadwalladr2017-aq}. L’entreprise a également pu acheter de nombreuses bases de données pour compléter ses modèles et ainsi parvenir à \say{capturer chaque aspect de l’environnement d’information de chaque électeur} \cite{Hern2018-ae}.\\

Viens alors le profilage psychologique des personnes, selon le modèle du Big Five, modèle en psychologie qui se propose de décrire la personnalité des individus selon 5 grands axes \cite{Rothmann2003-gd}. L’objectif est de pouvoir déterminer si la personne considérée est favorable ou non au camps que soutient CA. Il devient alors possible de mettre en place un micro-targeting, c’est-à-dire un ciblage publicitaire personnalisé passant notamment par des publicités sur les réseaux sociaux \cite{Sanjaume2010-ys}. L'objectif étant de faire changer de camps les profils déterminés comme tels, mais aussi de pousser les opposants à la dissension, et au vote blanc.\\

Il est cependant difficile de mesurer l’impact réel de ces campagne de manipulations (PSYOPS). Le travail de recherche de nombreux chercheurs, parmi lesquels Sara Suárez-Gonzalo, permet cependant d’y voir un peu plus clair sur les enseignements à tirer de cette affaire. Pour elle, le cas de CA durant la campagne présidentielle américaine de 2016 n’est que la conséquence inévitable des possibilités offertes par la technologie et le manque d’encadrement dans le milieu de la protection et de l’exploitation des données personnelles. Même si elle rappelle que l’influence réelle de CA est difficile à mesurer et ne doit pas être exagéré, elle souligne que les technique utilisées, par exemple le micro-targeting \say{s'inscrit dans un phénomène plus large de manipulation des médias lié aux nouvelles technologies et à l'effet de bulle qui s'est avéré être une raison importante de la montée de la désinformation en ligne et de la radicalisation des idées et opinions politiques.} \cite{Suarez-Gonzalo2018-ct}.\\

De Donald Trump jusqu’au Brexit, de Cambridge Analytica à Emerdata en passant par Palantir, les sociétés spécialisées dans le domaine de l’analyse et de la manipulation comportementale à grande échelle se multiplient. Il est réellement effrayant de voir quelles dérivent la collecte massive de données personnelles peut avoir comme impact sur nos sociétés, particulièrement dans le contexte de l’exercice de la démocratie. Même si de nouvelles réglementations tel que le RGPD ouvrent de nouvelles voies dans le protection des données personnelles, il me semble primordial de continuer à légiférer dans ce sens afin de protéger l’avenir de nos démocratie et de nos vies privées.\\

\newpage

\printbibliography[
heading=bibintoc,
title={Bibliographie \& Références}
]







\restoregeometry

\end{document}

