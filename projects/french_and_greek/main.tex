\documentclass[12pt]{article} % text width
\usepackage[utf8]{inputenc} % encode text to utf8
\usepackage[T1]{fontenc} % use T1 font encoding for french

% paragraph formatting: https://www.overleaf.com/learn/latex/Paragraph_formatting
\setlength{\parindent}{1em}
\setlength{\parskip}{1em}


\usepackage[greek, frenchb]
{babel} % text correction
\newcommand*{\textingreek}[1]{%
	\foreignlanguage{greek}{#1}%
} % new command for greek text
\newcommand*{\tig}[1]{\textingreek{#1}} %

\title{Exemple de texte avec du texte en grec}
\author{Florian Rascoussier}
\date{March 2022}

\begin{document}

\maketitle

\section{Programmation et code}
La programmation est un terme générique. Composé du suffixe \textit{-ation}, du latin \textit{-atio}, utilisé pour signifier un action, et du nom \textit{programme} lui-même issue via le latin \textit{programma} du grec ancien \textgreek{pr'ogramma}.
qui peut désigner divers concepts selon le domaine, par exemple le terme \textit{programmation} désigne, dans le milieu du cinéma, l'action de déterminer les programmes ou films d'une salle donnée.

\end{document}